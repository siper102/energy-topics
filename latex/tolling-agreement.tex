\documentclass{article}

\usepackage{amsmath, amssymb, amsfonts}
\usepackage{geometry}
\usepackage{booktabs}
\usepackage{hyperref}
\usepackage{tcolorbox}
\usepackage{enumitem}

\geometry{a4paper, margin=1in}

\title{\textbf{Arbitrage-Free Valuation of a Gas-Fired Tolling Agreement}}
\author{Model Specification}
\date{\today}

\begin{document}

\maketitle

% =====================================================
\section{Objective}

A gas-fired tolling agreement grants the holder the right, but not the obligation, 
to convert natural gas into electricity at a specified power plant.
Economically, it represents a \textbf{strip of American-style call options on the spark spread},
with path dependency induced by start-up costs and unit-level dispatch constraints.

% =====================================================
\section{Market Characteristics}

The valuation relies on two underlying commodities with distinct physical properties:

\begin{description}
    \item[Natural Gas ($G_t$):] A storable commodity. Its price dynamics are driven by storage levels, supply/demand, and weather. The forward curve is derived from liquidly traded instruments.
    \item[Electricity ($P_t$):] A non-storable commodity (flow). Since it cannot be economically stored in large quantities, prices must clear the market instantaneously, leading to high volatility and spikes.
\end{description}

The \textbf{Power Spot Price} ($S_t$) represents the price for immediate delivery. In an arbitrage-free framework, the spot price is the limit of the forward price as time to maturity approaches zero:
\[
S_t := F_P(t,t) = \lim_{T \to t} F_P(t, T).
\]

% =====================================================
\section{Risk-Neutral Measure and No-Arbitrage}

All valuation is performed under the risk-neutral measure $\mathbb Q$. 

\subsection{The Martingale Condition: Aligning with the Market}

The fundamental requirement of "No-Arbitrage" is that the model must be consistent with the current market prices of liquid assets (Forward contracts). Mathematically, this requires that the discounted asset prices are \textbf{martingales}.

In practical terms, this means the \textbf{expected value} of the future spot price in our model must exactly equal the current market \textbf{forward price}:
\[
\mathbb E^{\mathbb Q}[\text{Spot}_t] = \text{Forward}(0,t)
\]

\textbf{Why is this necessary?}
\begin{itemize}
    \item The Forward curve represents the market's consensus on fair value for future delivery.
    \item If our model predicted an average future spot price of \$50 while the market forward was trading at \$40, it would imply a "free lunch" (arbitrage) exists: one could buy the forward at \$40 and statistically expect to sell it at \$50.
    \item To prevent this, we mathematically "force" (calibrate) the drift of our Gas and Power simulations so they center exactly on the forward curves.
\end{itemize}

% -----------------------------------------------------
\subsection{Natural Gas Dynamics under $\mathbb Q$}

To satisfy the condition $\mathbb E^{\mathbb Q}[G_t] = F_G(0,t)$, we define the dynamics as:

\begin{equation}
dG_t = \mu_{\mathrm{fwd}}(t) G_t dt + \sigma_g G_t dW_t^{\mathbb Q,g},
\quad
\mu_{\mathrm{fwd}}(t) = \frac{\partial}{\partial t} \ln F_G(0,t)
\end{equation}

This results in the solution:
\begin{equation}
    G_{t} = F_G(0, t) \exp\left(-\frac{1}{2} \sigma_{g}^{2} t + \sigma_{g} W_t^{\mathbb Q,g}\right)
\end{equation}
Notice that $F_G(0,t)$ appears explicitly in the equation, ensuring the model is anchored to the market curve.

% -----------------------------------------------------
\subsection{Power Dynamics under $\mathbb Q$ (MRJD with Compensation)}

Similarly for power, we require $\mathbb E^{\mathbb Q}[P_t] = F_P(0,t)$. The price is modeled as the Forward curve multiplied by a stochastic factor:
\begin{equation}
P_t = F_P(0,t)\exp(X_t),
\qquad X_0 = 0
\end{equation}

The stochastic factor $X_t$ (Mean-Reverting Jump Diffusion) is defined as:
\begin{equation}
\boxed{
\begin{aligned}
dX_t = {} & -\kappa X_t dt
+ \sigma_p dW_t^{\mathbb Q,p}
+ J dN_t \\
& - \lambda\left(\mathbb E[e^{J}] - 1\right)dt
\end{aligned}
}
\end{equation}

\textbf{The Compensator Term:}
The last term, $- \lambda(\mathbb E[e^{J}] - 1)dt$, is a "correction" drift. It counteracts the upward pressure caused by the jumps ($J$), ensuring that the expected value of the stochastic factor $X_t$ remains neutral ($\mathbb E[e^{X_t}]=1$). Without this, the presence of price spikes would artificially inflate the model's average price above the forward curve.

% -----------------------------------------------------
\subsection{Correlation Structure}

\begin{equation}
dW_t^{\mathbb Q,g} \, dW_t^{\mathbb Q,p} = \rho \, dt
\end{equation}

% =====================================================
\section{Physical Measure ($\mathbb P$)}

The physical measure describes real-world price evolution and is used for risk management (e.g., VaR). It differs from $\mathbb Q$ because it includes the risk premium.

\subsection{Gas}
\begin{equation}
dG_t = \mu_{\mathrm{real}} G_t dt + \sigma_g G_t dW_t^{\mathbb P,g}
\end{equation}

\subsection{Power}
\begin{equation}
d(\ln P_t) =
\kappa(\theta_{\mathrm{LRMC}}(t) - \ln P_t)dt
+ \sigma_p dW_t^{\mathbb P,p}
+ J dN_t^{\mathbb P}
\end{equation}
Under $\mathbb P$, prices revert to the \textbf{Long Run Marginal Cost (LRMC)}, not the Forward curve.

% =====================================================
\section{Plant Representation: Multiple Units}

The plant consists of $N_{\mathrm{units}}$ independent units:

\begin{center}
\begin{tabular}{cccc}
\toprule
Unit & Heat Rate & Capacity & Start Cost \\
\midrule
$i$ & HR$_i$ & Cap$_i$ & $K_{\mathrm{start},i}$ \\
\bottomrule
\end{tabular}
\end{center}

\begin{description}
    \item[Heat Rate ($\mathrm{HR}_i$):] Efficiency (MMBtu/MWh). Lower is better.
    \item[Capacity ($\mathrm{Cap}_i$):] Max output (MW).
    \item[Start Cost ($K_{\mathrm{start},i}$):] Cost to turn the unit on (EUR).
\end{description}

% =====================================================
\section{Spark Spread Payoff}

The \textbf{Spark Spread} is the gross margin of the plant. For unit $i$ at hour $h$:

\begin{equation}
\pi_{h,i}
=
\underbrace{\left(
P_{t_h}
-
\mathrm{HR}_i G_{t_h}
\right)}_{\text{Unit Margin}}
\cdot \mathrm{Cap}_i
\end{equation}
The Unit Margin (€) gives us the difference between what we spend per MwH and what we earn per MwH so it is the profit/loss for this unit at hour h.
% =====================================================
\section{Optimal Dispatch Problem}

The operator maximizes daily profit against start costs:

\begin{equation}
\text{Daily Value}_D =
\sum_{i=1}^{N_{\mathrm{units}}}
\max
\left\{
\sum_{h=1}^{24}\pi_{h,i}
-
K_{\mathrm{start},i}, 0
\right\}
\end{equation}

This formula ensures that we start a unit whenever it is profitable to start it against the startup costs. 
This also means that we run the unit all day when it is expected to be profitable even when there are hours where we sell with a loss.
% =====================================================
\section{Tolling Agreement Valuation}

\begin{equation}
V_0 =
\sum_{D=1}^{365}
e^{-rT_D}
\mathbb E^{\mathbb Q}\left[\text{Daily Value}_D\right]
\end{equation}

% =====================================================
\section{Monte Carlo Valuation Algorithm}

\begin{tcolorbox}
\begin{enumerate}
    \item Simulate correlated $(G_t,X_t)$ paths under $\mathbb Q$, ensuring expected values match forwards.
    \item Construct $P_t = F_P(0,t)e^{X_t}$.
    \item Compute hourly spark spreads $\pi_{h,i}$.
    \item Apply dispatch and start logic (Real Option).
    \item Average discounted cashflows.
\end{enumerate}
\end{tcolorbox}

\end{document}