\documentclass{article}

\usepackage{amsmath, amssymb, amsfonts}
\usepackage{geometry}
\usepackage{booktabs}
\usepackage{hyperref}
\usepackage{tcolorbox}

\geometry{a4paper, margin=1in}

\title{\textbf{Arbitrage-Free Valuation of a Gas-Fired Tolling Agreement}}
\author{Model Specification}
\date{\today}

\begin{document}

\maketitle

% =====================================================
\section{Objective}

A gas-fired tolling agreement grants the holder the right, but not the obligation, 
to convert natural gas into electricity at a specified power plant.
Economically, it represents a \textbf{strip of American-style call options on the spark spread},
with path dependency induced by start-up costs and unit-level dispatch constraints.

% =====================================================
\section{Market Characteristics}

\begin{itemize}
    \item \textbf{Natural Gas ($G_t$):} Storable commodity; forward curve derived from traded instruments.
    \item \textbf{Electricity ($P_t$):} Non-storable commodity; forward curve is primitive.
\end{itemize}

The power spot price is defined consistently as:
\[
S_t := F_P(t,t).
\]

% =====================================================
\section{Risk-Neutral Measure and No-Arbitrage}

All valuation is performed under the risk-neutral measure $\mathbb Q$.
No-arbitrage requires that traded forward prices are martingales.

% -----------------------------------------------------
\subsection{Natural Gas Dynamics under $\mathbb Q$}

\begin{equation}
dG_t = \mu_{\mathrm{fwd}}(t) G_t dt + \sigma_g G_t dW_t^{\mathbb Q,g},
\quad
\mu_{\mathrm{fwd}}(t) = \frac{\partial}{\partial t} \ln F_G(0,t)
\end{equation}

Ensuring:
\[
\mathbb E^{\mathbb Q}[G_t] = F_G(0,t).
\]
This gives us:
\begin{equation}
    G_{t} = F(0, t) \exp(-\frac{1}{2} \sigma_{g}^{2} t + \sigma_{g} W_t^{\mathbb Q,g})
\end{equation}
% -----------------------------------------------------
\subsection{Power Dynamics under $\mathbb Q$ (MRJD with Compensation)}

The power price is modeled as:
\begin{equation}
P_t = F_P(0,t)\exp(X_t),
\qquad X_0 = 0
\end{equation}

where the stochastic factor $X_t$ follows:
\begin{equation}
\boxed{
\begin{aligned}
dX_t = {} & -\kappa X_t dt
+ \sigma_p dW_t^{\mathbb Q,p}
+ J dN_t \\
& - \lambda\left(\mathbb E[e^{J}] - 1\right)dt
\end{aligned}
}
\end{equation}

with:
\begin{itemize}
    \item $N_t$: Poisson process with intensity $\lambda$
    \item $J \sim \mathcal N(\mu_J, \sigma_J^2)$
    \item $\mathbb E[e^{J}] = \exp(\mu_{J} + \frac{1}{2} \sigma_{J}^{2})$
\end{itemize}

Since:
\[
\mathbb E[e^{J}] = \exp\left(\mu_J + \tfrac12\sigma_J^2\right),
\]
the compensator ensures:
\[
\mathbb E^{\mathbb Q}[P_t] = F_P(0,t),
\]
thus preserving no-arbitrage.

% -----------------------------------------------------
\subsection{Correlation Structure}

\begin{equation}
dW_t^{\mathbb Q,g} \, dW_t^{\mathbb Q,p} = \rho \, dt
\end{equation}

% =====================================================
\section{Physical Measure ($\mathbb P$)}

The physical measure describes real-world price evolution and is used for risk management.

\subsection{Gas}

\begin{equation}
dG_t = \mu_{\mathrm{real}} G_t dt + \sigma_g G_t dW_t^{\mathbb P,g}
\end{equation}

\subsection{Power}

\begin{equation}
d(\ln P_t) =
\kappa(\theta_{\mathrm{LRMC}}(t) - \ln P_t)dt
+ \sigma_p dW_t^{\mathbb P,p}
+ J dN_t^{\mathbb P}
\end{equation}

Jump intensities and diffusion drifts may differ between $\mathbb P$ and $\mathbb Q$.

% =====================================================
\section{Plant Representation: Multiple Units}

The plant consists of $N_{\mathrm{units}}$ independent units:

\begin{center}
\begin{tabular}{cccc}
\toprule
Unit & Heat Rate & Capacity & Start Cost \\
\midrule
$i$ & HR$_i$ & Cap$_i$ & $K_{\mathrm{start},i}$ \\
\bottomrule
\end{tabular}
\end{center}

% =====================================================
\section{Spark Spread Payoff}

For unit $i$ at hour $h$:

\begin{equation}
\pi_{h,i}
=
\left(
P_{t_h}
-
\mathrm{HR}_i G_{t_h}
\right)\cdot \mathrm{Cap}_i
\end{equation}

% =====================================================
\section{Optimal Dispatch Problem}

Daily dispatch is obtained by:

\begin{equation}
\text{Daily Value}_D =
\sum_{i=1}^{N_{\mathrm{units}}}
\max
\left\{
\sum_{h=1}^{24}\pi_{h,i}
-
K_{\mathrm{start},i}, 0
\right\}
\end{equation}

A merit-order heuristic is applied for tractability.

% =====================================================
\section{Tolling Agreement Valuation}

\begin{equation}
V_0 =
\sum_{D=1}^{365}
e^{-rT_D}
\mathbb E^{\mathbb Q}\left[\text{Daily Value}_D\right]
\end{equation}

% =====================================================
\section{Monte Carlo Valuation Algorithm}

\begin{tcolorbox}
\begin{enumerate}
    \item Simulate correlated $(G_t,X_t)$ paths under $\mathbb Q$.
    \item Construct $P_t = F_P(0,t)e^{X_t}$.
    \item Compute hourly spark spreads for each unit.
    \item Apply dispatch and start logic.
    \item Average discounted cashflows.
\end{enumerate}
\end{tcolorbox}

\end{document}
